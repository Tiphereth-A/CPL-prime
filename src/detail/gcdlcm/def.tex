\begin{frame}[fragile]{定义}
	\only<1->{\begin{definition}[最大公约数]
			\label{prime:def:gcd}

			对于两个\textbf{不全为零}的整数\(a,b\), 最大公约数即为\(a,b\)中最大的正公因数, 记作\((a,b)\), 或者\(\gcd(a,b)\)
		\end{definition}}

	\only<2->{\begin{definition}[最小公倍数]
			\label{prime:def:lcm}

			对于两个\textbf{不全为零}的整数\(a,b\), 最小公倍数即为\(a,b\)中最小的正公倍数, 记作\([a,b]\), 或者\(\lcm(a,b)\)
		\end{definition}}
\end{frame}


\begin{frame}[fragile]{定义}
	写成式子就是

	\begin{equation}
		\label{prime:eq:gcd}
		(a,b):=\max\{d\in\Bbb{N}^*:d\mid a,d\mid b\}
	\end{equation}

	\begin{equation}
		\label{prime:eq:lcm}
		[a,b]:=\min\{d\in\Bbb{N}^*:a\mid d,b\mid d\}
	\end{equation}
\end{frame}


\begin{frame}[fragile]{定义}
	这一组概念自然可以推广到多个整数之间

	\begin{equation}
		\label{prime:eq:gcdn}
		(a_1,a_2,\dots,a_n):=\max\{d\in\Bbb{N}^*:d\mid a_i,~i=1,2,\dots,n\}
	\end{equation}

	\begin{equation}
		\label{prime:eq:lcmn}
		[a_1,a_2,\dots,a_n]:=\min\{d\in\Bbb{N}^*:a_i\mid d,~i=1,2,\dots,n\}
	\end{equation}
\end{frame}
